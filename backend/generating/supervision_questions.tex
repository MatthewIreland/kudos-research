\begin{enumerate}

\question{}
\item \textbf{Explain the notion of an Inverse Problem and how computer vision can be regarded thereby in a formal sense as inverse graphics}
% enter text here
\question{}
\item \textbf{Why are many problems in computer vision ill-posed? In general, what metaphysical assumptions may be invoked by a vision algorithm in order to make an inference task well-posed, and thereby make computations that would otherwise be impossible, possible?}
% enter text here
\question{}
\item \textbf{Contrast the use of linear versus non-linear operators in computer vision, giving at least one example of each. What can linear operators accomplish, and what are their fundamental limitations? With non-linear operators, what heavy price must be paid and what are the potential benefits?}
% enter text here
\question{}
\item \textbf{Using the second finite difference operator [-1, 2, -1] for edge detection in an image, show how the pixel values in the row of the image given below are changed by discrete convolution with this operator: 
 [...,0, 0, 0, 0, 5, 5, 5, 5, 5, 0, 0, 0, 0, ...] 
 Explain what modifications to your approach would be required if}
\begin{automarkable}{ocaml}{www.automarkable.com}
\end{automarkable}
\begin{enumerate}
\question{}
\item \textbf{The filter kernel were asymmetric}
% enter text here
\question{}
\item \textbf{The filtering operation were applied to the whole row of the image - how do you treat the first and second pixels in the row?}
% enter text here
\end{enumerate}
\question{}
\item \textbf{}
% enter text here
\begin{enumerate}
\question{}
\item \textbf{Explain the operation of the 2D Fourier Transform and its applications in computer vision. Construct examples to aid your answer.}
% enter text here
\question{}
\item \textbf{Compare and contrast the 2D Fourier Transform with the 2D wavelet transform.}
% enter text here
\end{enumerate}

\end{questions}
\end{enumerate}